\documentclass[12pt]{article}
\usepackage{url,graphicx,tabularx,array}
\usepackage[margin=1in]{geometry}
\setlength{\parskip}{1ex} %--skip lines between paragraphs
\setlength{\parindent}{0pt} %--don't indent paragraphs

\usepackage{algorithmic}
\usepackage{algorithm}
\usepackage{ amssymb }
\usepackage{ latexsym }
\usepackage{ amsmath }
\usepackage{ amsthm }
%-- Commands for header
\renewcommand{\title}[1]{\textbf{#1}\\}
\renewcommand{\line}{\begin{tabularx}{\textwidth}{X>{\raggedleft}X}\hline\\\end{tabularx}\\[-0.5cm]}
\newcommand{\leftright}[2]{\begin{tabularx}{\textwidth}{X>{\raggedleft}X}#1%
& #2\\\end{tabularx}\\[-0.5cm]}

\newtheorem{defn}{Definition}[section]
\newtheorem{conjecture}{conjecture}[section]
\newtheorem{lemma}{Lemma}[section]
\newtheorem{corollary}{Corollary}[section]
\newtheorem{question}{Question}[section]
\newtheorem{proposition}{Proposition}[section]


%\linespread{2} %-- Uncomment for Double Space
\begin{document}

\title{Homework 1: CMPS 242}
\line
\leftright{\today}{Bryan Matsuo (bmatsuo@soe.ucsc.edu) \& John St. John (jstjohn@soe.ucsc.edu)} %-- left and right positions in the header
\begin{enumerate}
\item \textbf{Version spaces: }

\begin{enumerate}
\item There is no inductive bias in the hypothesis class. The hypotheses predicting a new instance are unconstrained by the previously seen instances.
\item 
Let $V_m$ be the version space after seeing $m$ different (noise-free) labeled examples $x_1, \dots, x_m$. For an unseen instance $x_{m+1}$, exactly half of the version space predicts ``$+$'', and half the version space predects ``$-$'' as a label for $x_{m+1}$.

\begin{proof}
    After seeing $m$ instances and their labels, the set of consistent hypothesis is the set of all labelings for the elements of $X$ that are consistent with the first $m$ instances seen. Since the labelings of the $n-m$ unseen instances are unconstrained by the seen ones, for a given element of $X$ that has not yet been seen there are exactly $2^{n-m-1}$ hypothesis labeling the element ``$+$'', and the same number labeling the element ``$-$''.
\end{proof}

\end{enumerate}
\item \textbf{ Learning disjunctions: }

\begin{enumerate}
\item A general hypothesis is one that assumes that any unobserved point
that could consistently be classified as a positive case, is classified
that way. In the base case this will classify all points as positive.
Any observed positive example will not change our general hypothesis
space. However a negative example will force us to prune away instance
assignments from our general hypothesis. See Algorithm~\ref{Alg:prune} for pseudo code.
\begin{algorithm}
\caption{Efficient training algorithm: prunes all instances in negative examples from the hypothesis. }
\label{Alg:prune}
\begin{algorithmic}
\STATE $\mathcal{H} \gets a_1=0 \vee a_1=1 \vee \ldots \vee a_n=0 \vee a_n=1$
\FORALL {$X \in \textnormal{Train}$}
	\IF {$X \in F$}
		\FORALL {$a_i=x\in X$}
			\STATE $\mathcal{H} \gets \mathcal{H} \setminus \left\{a_i=x\right\}$
		\ENDFOR
	\ENDIF
\ENDFOR
\end{algorithmic}
\end{algorithm}

\item No the most specific hypothesis in this case is not well defined.
The general algorithm for this would be to ignore negative instances,
and for each positive observation insert the minimal set of instance
assignments that properly classifies the set of positive observations.
Consider a simple case with one positive observation that has a set
of two instance assignments $\left\{ a_{1}=1,a_{2}=0\right\} $. In
this case we have three equally good specific hypotheses, $h_{1}=\left[a_{1}=1\right]$
, $h_{2}=\left[a_{2}=0\right]$, and $h_{3}=\left[a_{1}=1\vee a_{2}=0\right]$.
By definition this is not well defined.
\end{enumerate}


\item \textbf{VC Dimension (242 version): }

Consider the hypothesis class of homogeneous half-spaces in $\mathcal{R}^d$ where the instance vectors are in $\mathcal{R}^d$ and the hypothesis class $H$ is the set of all $h_w$ having the form $h_w(x) = +1$ if $w \cdot x > 0$, and $-1$ otherwise.
Determine the VC-dimension of homogeneous half-spaces in $\mathcal{R}^d$ as a function of $d$.

This solution will make use of the following proposition provided as a hint.
\begin{proposition}
    \label{prop:lin-comb}
    For any set $S$ of $d+1$ points in $\mathcal{R}^d$, there is at least one point $x \in S$ that can be expressed as a linear combination of the other points in $S$.
\end{proposition}

\begin{lemma}
    The VC-dimension of homogeneous half-spaces in $\mathcal{R}^d$ is at least $d$.
\end{lemma}

\begin{proof}
    Consider the set of instances $X_d = \{e_1, e_2, \dots, e_d\}$ where $e_i$ is the standard $i$-th basis vector in $\mathcal{R}^d$ (component $i$ of vector $e_i$ is $1$ and the rest are $0$).
    Let $y_1, \dots, y_d \in \{-1,+1\}$ be a labeling of $X_d$ where $e_i$ recieves label $y_i$.

    Consider the vector $w = (y_1, \dots, y_d)$.
    Notice that $w \cdot e_i = y_i$.
    Thus, the hypothesis $h_w$ that predicts using vector $w$ will consistently predict all the labels of instances in $X_d$.

    Due to the arbitrary labels on the instance set $X_d$, we can conclude that the hypothesis class $H$ shatters set $X_d$ and that the VC demension of the hypothesis class is at least $d$.
\end{proof}

\begin{lemma}
    The VC-dimension of homogeneous half-spaces in $\mathcal{R}^d$ is at most $d$.
\end{lemma}

\begin{proof}
    Let $V=(v_0, v_1, \dots, v_d)$ be any set of $d+1$ points in $\mathcal{R}^d$.
    From proposition \ref{prop:lin-comb}, we know that there is a vector $v_k$ in $V$ which can be expressed as a linear combination of the other points in $V$.
    Without loss of generality assume that $v_0$ can be expressed as a linear combination of $v_1, \dots, v_d$.
    That is,
    \[
    v_0 = \sum_{i=1}^{d} a_i v_i
    \].

    Define a labeling $y_k$ of point $v_k \in V$ in the following way.
    \[
    y_k = \begin{cases}
        +1  &  \text{if } k = 0   \\
        -1  &  \text{if } a_k > 0 \\
        +1  &  \text{if } a_k \leq 0 \\
    \end{cases}
    \].

    Let $w$ be a vector in $\mathcal{R}^d$ and assume that hypothesis $h_w$ is consistent with the labels $y_1, \dots, y_d$.
    Due to the construction of the labeling, this implies that $a_k (w \cdot v_k) \leq 0$ for $1 \leq k \leq d$.
    Consider now the prediction that $h_w$ makes for instance $v_0$.
    \begin{align*}
        w \cdot v_0 &= w \cdot \bigg( \sum_{i=1}^{d} a_i v_i \bigg) \\
            &= \sum_{i=1}^{d} a_i (w \cdot v_i) \\
            &\leq 0
    \end{align*}
    But the label of $v_0$ is defined as $y_0 = +1$.
    So no hypothesis $h_w$ can be consistent with the labeling $y_0, \dots, y_d$, and thus that the hypothesis class can not shatter the set instance set $V$.
    Because $V$ is an arbitrary set of $d+1$ instances, we have thus shown that the VC-dimension of the hypothesis class is no more than $d$.
\end{proof}

\begin{lemma}
    The VC-dimension of homogeneous half-spaces in $\mathcal{R}^d$ is $d$.
\end{lemma}
\begin{proof}
    This is the direct result of combining the previous two lemmas.
\end{proof}

\item \textbf{Weka: }

\begin{enumerate}
\item To seperate out Iris-setosa from the other classes, there is one attribute,
peta-length, that looks like it results in perfect seperation by itself
from examining the histogram.
\item The trimmed tree also has the petal-length atribute as the top node
of the decision tree. This is a good sign.
\item It looks like petal length or petal width are enough in 1 dimention
to seperate out Iris-setosa from the others with high confidence.
This can be clearly seen in the histogram for petal-length, but the
seperation is a little too close to see the clear divide on the histogram
for petal-width. Choosing a smaller bin size for the petal-width histogram
would solve this problem, and it is obvious when examining the points
under the visualize tab.

\begin{enumerate}
\item Side note: Seperating the remaining two classes of irises does not
look possible with a single dimension. In 2d petal-length to petal-width
looks like its enough to seperate out all three classes in the majority
of cases with possibly a few false positives. Indeed the J48 decision
tree found a fine grained descriminating set of rules that are only
dependent on those two variables with around a 2\% FP rate.\end{enumerate}
\end{enumerate}
\end{enumerate}

\end{document}
