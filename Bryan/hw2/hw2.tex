\documentclass[11pt]{article}

\usepackage{fullpage,latexsym,amsthm,amsmath,algorithmic,graphicx,subfigure,color,verbatim,epsfig,fancyhdr,multicol}
%\usepackage[normalmargins]{savetrees}

% Define new commands.
\newtheorem{defn}{Definition}[section]
\newtheorem{conjecture}{conjecture}[section]
\newtheorem{lemma}{Lemma}[section]
\newtheorem{corollary}{Corollary}[section]
\newtheorem{question}{Question}[section]
\newtheorem{proposition}{Proposition}[section]

\newcommand{\eqnref}[1]{(\ref{eqn: #1})}
\newcommand{\figref}[1]{(\ref{fig: #1})}
\newcommand{\algref}[1]{(\ref{alg: #1})}
\newcommand{\tabref}[1]{(\ref{tab: #1})}

%\newcommand{\abcitem}{\item[\alph({alphaenum})]\stepcounter{alphaenum}}
\newenvironment{abclist}{
\begin{list}{(\alph{alphaenum}) ~}
    {\newcounter{alphaenum} \setcounter{alphaenum}{1} \usecounter{alphaenum} }
    }
    {
\end{list}
}


\begin{document}

% Make the document title.
\title{CMPS 242 - Spring 2011 - Homework assignment \#2}
\author{ {Bryan Matsuo}\\
{Department of Computer Science, UCSC}}
%\date{\today}
\maketitle

%\begin{multicols}{2}
\section*{1 Number Crunching\dots}

\section*{2 Naive Bayes}

\section*{3 Bayesian Decision Theory}

In this problem we are classifying automobiles.
Given an instance.
We can predict {\bf B}us, {\bf C}ar, or a {\bf T}ruck.
We may also choose to {\bf A}bstain (give no prediction).
The cost matrix for the decisions is
\begin{center}
    \begin{tabular}{|l|c|c|c|c|}
        \hline
        loss &
        \multicolumn{4}{|c|}{prediction} \\
        \multicolumn{1}{|r|} {class} & B & C & T & A \\
        \hline
        B & 0 & 1 & 1 & $a$ \\
        C & 1 & 0 & 1 & $a$ \\
        T & 1 & 1 & 0 & $a$ \\
        \hline
    \end{tabular}
\end{center}
where $a$ is a cost (or loss) incurred from abstaining from a prediction.
We will denote the loss of a prediction $\hat y$ given a true label $y$ as
$L(\hat{y} \mid y)$. For example, $L(B|C)=1, L(T|T) = 0 \text{ and } L(A|C)=a$.

\begin{abclist}
\item Assume that for an arbitrary instance $P(B) = 0.5, P(C) = 0.25, \text{ and } P(T) = 0.25$.
    The risk $R(\hat y)$ of any prediction $\hat y$ is 
    \begin{equation}
        R(\hat y) = E_y[L(\hat{y} \mid y)] = \sum_{y \in \left\{ B,C,T \right\}} P(y) L(\hat{y} \mid y)
        \label{eqn: risk}
    \end{equation}.

    Equation \eqnref{risk} allows us to calculate the risk associated with each prediction (or lack there of)
    \begin{center}
        \begin{tabular}{|l|c|}
            \hline
            class & risk \\
        \hline
        B & 0.5 \\
            C & 0.75 \\
            T & 0.75 \\
            A & $a$ \\
            \hline
        \end{tabular}
    \end{center}.

\item A risk averse predictor would predict {\bf B} for any new instance if $a > 0.5 = R(B)$.
    They could predict {\bf B} or {\bf A}bstain with equal expectation for cost if $a = 0.5$.
    As $a$ tends to zero, the best predition for the risk averse predictor becomes to always abstain from giving a prediction, for which the cost tests to zero.
\end{abclist}


\section*{4 Expectation of a product}
\begin{lemma}[Product of idependent variables]
    Let $V$ and $W$ be independent random variables. Then $E[VW] = E[V]E[W]$.
\end{lemma}
\begin{proof}
    Let $Z=VW$.
    The joint probability density of $V$ and $Z$ is given as
    \begin{equation}
        p(V=v,Z=z) = p(V=v)p(Z=z|V=v) = p(V=v)p\bigg(W=\frac{z}{v}\bigg)
    \end{equation}
    Then the probability density of $Z$, $p(Z=z)$ is
    \begin{eqnarray}
        p(Z=z) &=& \int_{-\infty}^{\infty}p(V=v,Z=z)dv \\
        &=& \int_{-\infty}^{\infty}p(V=v)p\bigg(W=\frac{z}{v}\bigg)dv
        \label{eqn: indep_prod_density}
    \end{eqnarray}
    The expected value of $Z$ is then
    \begin{eqnarray}
        E[Z] &=& \int_{-\infty}^{\infty} zp(Z=z) dz \nonumber \\
            &=& \int_{-\infty}^{\infty} \int_{-\infty}^{\infty}z p(V=\nu)p\bigg(W=\frac{z}{\nu}\bigg)d\nu dz \\
        \label{eqn: full_expected_indep_prod}
    \end{eqnarray}
    Making the change of variables $\nu = v, z = vw$, we get
    \begin{eqnarray}
        E[Z] &=& \int_{-\infty}^{\infty} \int_{-\infty}^{\infty}
            vwp(V=v)p(W=w) (v \text{ \color{red}{\dots shit}}) dv dw \\
            &=& \int_{-\infty}^{\infty} wp(W=w) \int_{-\infty}^{\infty} v^2 p(V=v) dv dw \\
            &=& E[W]E[V^2]
    \end{eqnarray}
    That extra $v$ term that popped out of the determinant of the Jacobian really throws things off.
    I notice now that I never actually used the fact that $V$ and $W$ are independent.
    This could be the key flaw in the proof, but I can't see where to apply it.

    {\color{blue}Alternatively}, the conditional probability density of $Z$ can be viewed as a Dirac delta function.
    \begin{equation}
        p(Z=z|V=v,W=w) = \delta(z - vw)
        \label{eqn: p z}
    \end{equation}
    The delta function is shifted so all the probability \emph{mass} of $Z$($\in \mathcal{R}$) is centered over $vw$.
    That is, $p(Z=z|V=v,W=w)$ is equal to 0 for all $z \neq vw$ and $\int_{-\infty}^{\infty}p(Z=z|V=v,W=w)dz = \int_{-\infty}^{\infty}\delta(z-vw)dz = 1$.
    The final observation about the delta function that we will make is that 
    \begin{equation}
        \int_{-\infty}^{\infty} f(z)\delta(z-vw)dz = f(vw)
        \label{eqn: delta func eval}
    \end{equation}.
    In particular, $E[Z] = \int_{-\infty}^{\infty} z\delta(z-vw)dz = vw$.
    
    The joint probability density of $V$, $W$, and $Z$ can be written as
    \begin{eqnarray}
        p(V=v, W=w, Z=z) &=& p(V=v, W=w)p(Z=z|V=v,W=w) \nonumber \\
            &=& p(V=v)p(W=w)\delta(z - vw)
            \label{eqn: joint density}
    \end{eqnarray}.
    with equations \eqnref{p z}, \eqnref{delta func eval}, and \eqnref{joint density} we can calculate the expected value of $Z=VW$.
    \begin{eqnarray}
        E[Z]
        &=& \int_{-\infty}^{\infty} \int_{-\infty}^{\infty} \int_{-\infty}^{\infty} zp(V=v,W=w,Z=z) dzdwdv \nonumber \\
        &=& \int_{-\infty}^{\infty} \int_{-\infty}^{\infty} \int_{-\infty}^{\infty} zp(V=v)p(W=w)\delta(z - vw) dzdwdv \nonumber \\
        &=& \int_{-\infty}^{\infty} p(V=v) \int_{-\infty}^{\infty} p(W=w) \int_{-\infty}^{\infty} z\delta(z - vw) dzdwdv \nonumber \\
        &=& \int_{-\infty}^{\infty} vp(V=v) \int_{-\infty}^{\infty} wp(W=w) dw dv = E[V]E[W]
        \label{eqn: z_mean}
    \end{eqnarray}
    Booyakasha!
\end{proof}
Information about the Dirac delta function was mostly found on wikipedia, unfortunately.

% Add a bibliography.
%\bibliographystyle{apalike}
%\bibliography{project}

%\end{multicols}

\end{document}
