\documentclass[12pt]{article}
\usepackage{url,graphicx,tabularx,array}
\usepackage[margin=1in]{geometry}
\setlength{\parskip}{1ex} %--skip lines between paragraphs
\setlength{\parindent}{0pt} %--don't indent paragraphs

\usepackage{algorithmic}
\usepackage{algorithm}
\usepackage{ amssymb }
\usepackage{ latexsym }
\usepackage{ amsmath }
\usepackage{ amsthm }
%-- Commands for header
\renewcommand{\title}[1]{\textbf{#1}\\}
\renewcommand{\line}{\begin{tabularx}{\textwidth}{X>{\raggedleft}X}\hline\\\end{tabularx}\\[-0.5cm]}
\newcommand{\leftright}[2]{\begin{tabularx}{\textwidth}{X>{\raggedleft}X}#1%
& #2\\\end{tabularx}\\[-0.5cm]}

\newtheorem{defn}{Definition}[section]
\newtheorem{conjecture}{conjecture}[section]
\newtheorem{lemma}{Lemma}[section]
\newtheorem{corollary}{Corollary}[section]
\newtheorem{question}{Question}[section]
\newtheorem{proposition}{Proposition}[section]

%\linespread{2} %-- Uncomment for Double Space
\usepackage{Sweave}
\begin{document}

\title{Homework 2: CMPS 242}
\line
\leftright{\today}{Bryan Matsuo (bmatsuo@soe.ucsc.edu) \& John St. John (jstjohn@soe.ucsc.edu)} %-- left and right positions in the header
\begin{enumerate}
\item \textbf{General Probability: }

\begin{enumerate}
\item %(A)
The prior probability that each point is positive is the mean probability of the point under the hypotheses weighted by the prior of each hypothesis.
%<<>>=
%library("plyr")
%print(h<-matrix(data=c(0.8,0.25,0.5,0.8,0.75,0.5,0.2,0.75,0.5),nrow=3,ncol=3,dimnames=list(c("a","b","c"),c("h1","h2","h3")), byrow=TRUE ))
%print(p<-matrix(data=c(rep(1/2,3),rep(4/10,3),rep(1/10,3)),nrow=3,ncol=3,dimnames=list(c("a","b","c"),c("h1.p","h2.p","h3.p"))))
%print(priors<- apply(p*h,1,sum))
%@
\item %(B)
To calculate this value, I first make a matrix of the probability of our observations a+,b-,c+ conditioned on each hypothesis. I will call this array cond.prob.obs. Next calculate the product of the probabilities of each hypothesis in the matrix to get the likelihood of that hypothesis.
%<<>>=
%print(cond.prob.obs <- matrix(data=c(h['a',], 1-h['b',], h['c',]),nrow=3,ncol=3,dimnames=list(c("a+","b-","c+"),c("h1","h2","h3")),byrow=TRUE))
%print( h.cond <- apply(cond.prob.obs,2,prod) )
%@
The hypothesis with the greatest likelihood is $h_3$.

\item %(C)
$P(h|a=+,b=-,c=+)$ is what we want to calculate for this. By law of probability, by definition it is $\frac{P(H, a=+, b=-, c=+)}{P(a=+,b=-,c=+)}=\frac{P(h)P(a=+|h)P(b=-|h)P(c=+|h)}{\sum_{i=1}^3 P(h=h_i,a,b,c)}$ but you don't need to look at the bottom term since it is the same for all instances.
%<<>>=
%print(prior_h<- p[1,])
%posterior<- h.cond*prior_h
%print(posterior<-posterior/sum(posterior))
%
%@
So without normalizing and turning these into probabilities, we can still simply choose the largest of these, which is now $h_2$.

\item %(D)
We want to know $P(a'=+ | a=+,b=-,c=+)$. Again we can rewrite this in terms of things we can look up in the table. $=\frac{P(a'=+, a=+, b=-, c=+)}{P(a=+,b=-,c=+)} = P(a'=+, h=h_1,a=+,b=-,c=+)+P(a'=+,h=h_2,\ldots)+P(a'=+,h=h_3,\ldots)$ and normalize it with also calculating $a'=-$. Where you calculate $P(a'=+, h=h_1,a=+,b=-,c=+)$ like $P(h_1)P(a=+|h_1)P(a'=+|h_1)P(b=-|h_1)P(c=+|h_1)$.
%
%<<>>=
%posterior.rhack <- matrix(rep(posterior,3),nrow=3,byrow=TRUE)
%joint.cond.prob<-posterior.rhack * h
%print(mean.posterior.pr<-aaply(joint.cond.prob,1,sum))
%@

\item %{E}
%do parts B, C,D again
Since the empirically observed frequencies of each letter are the same as before, the maximum likelihood hypothesis wouldn't change. 

The maximum a'posteriori hypothesis after seeing as the observed data grows, the priors will be decreasingly important.
%<<>>=
%posterior_2 <- h.cond*h.cond*prior_h
%print(posterior_2 <-posterior_2/sum(posterior_2))
%@

Similarly the mean posterior probability of + for each point after seeing the data will change.

%<<>>=
%posterior_2.rhack <- matrix(rep(posterior_2,3),nrow=3,byrow=TRUE)
%joint.cond.prob.2<-posterior_2.rhack * h
%print(mean.posterior.pr<-aaply(joint.cond.prob.2,1,sum))
%@


\end{enumerate}
\item \textbf{Naive Bayes:}

 We generate instances in fields using the class. We assume independence between the fields. We have one distribution $P(AP|C=H)$ and $P(AP|C=N)$.  Note that $P(AP=Y|C=H)$ and $P(AP=N|C=H)$ sum to one, but not $P(AP|C=H)$ and $P(AP|C=N)$. We assume $P(GPA|C=H)=N(\mu_H,\sigma_H^2)$ and $P(GPA|C=N)=N(\mu_N,\sigma_N^2)$. We now have 5 independent distributions when you include the two priors. We also need to estimate prior distributions $P(C=H), P(C=N)$ and can just use empirical things. When you predict you have a new instance with an unknown class, known AP, known GPA. We calculate probability $P(C)P(AP|C)P(GPA|C)$ for each class and choose the one with the greatest probability. Naive bays just means that we can assume $P(AP,GPA|C) = P(AP|C)P(GPA|C)$ (ie GPA and AP are independent) which can be a lot easier to calculate. So to solve the problem we just set GPA to be a variable, set AP to true or false, and calculate the decision point for both of those cases separately.


\begin{Schunk}
\begin{Sinput}
> dat
\end{Sinput}
\begin{Soutput}
  GPA AP class
1 4.0  y     H
2 3.7  y     H
3 2.5  n     H
4 3.8  n     N
5 3.3  y     N
6 3.0  y     N
7 3.0  n     N
8 2.7  n     N
9 2.2  n     N
\end{Soutput}
\begin{Sinput}
> print(mean.gpa.by.class <- ddply(dat, .(class), colwise(mean, 
+     .(mean.GPA = GPA))))
\end{Sinput}
\begin{Soutput}
  class mean.GPA
1     H      3.4
2     N      3.0
\end{Soutput}
\begin{Sinput}
> print(sd.gpa.by.class <- ddply(dat, .(class), colwise(sd, .(sd.GPA = GPA))))
\end{Sinput}
\begin{Soutput}
  class    sd.GPA
1     H 0.7937254
2     N 0.5403702
\end{Soutput}
\begin{Sinput}
> n.n <- sum(dat$class == "N")
> n.h <- sum(dat$class == "H")
> pc.n <- n.n/(n.n + n.h)
> pc.h <- n.h/(n.n + n.h)
> n.ap.y.class.h <- sum(dat$class == "H" & dat$AP == "y")
> n.ap.y.class.n <- sum(dat$class == "N" & dat$AP == "y")
> print(p.ap.y_class.h <- n.ap.y.class.h/n.h)
\end{Sinput}
\begin{Soutput}
[1] 0.6666667
\end{Soutput}
\begin{Sinput}
> p.ap.n_class.h <- 1 - p.ap.y_class.h
> print(p.ap.y_class.n <- n.ap.y.class.n/n.n)
\end{Sinput}
\begin{Soutput}
[1] 0.3333333
\end{Soutput}
\begin{Sinput}
> p.ap.n_class.n <- 1 - p.ap.y_class.n
\end{Sinput}
\end{Schunk}

So we need to calculate $P(AP=y|C=H)P(GPA|C=H) == P(AP=y|C=N)P(GPA|C=N)$ Using wolfram alpha to solve the equations on the following page we get the following rules:

Solving this equation gives two GPA cutoffs for prediction of H. That defines the interval where you would predict that a student will be normal. 

If AP courses are taken, predict H if the GPA is not between 1.99183 and 2.30188 \\
if AP courses are not taken, never predict H.

In reality though it doesn't make sense that a student with a GPA between 0 and 1.99 would be an honors student if they simply took an AP class. With this prior expectation we can just look at the upper portion of the predicted interval for H and say if a student gets a GPA between 4.0 and 2.3 then we predict H giving us the following alternate rule:

If AP courses are taken, predict H if the GPA is above 2.30188 \\
if AP courses are not taken, never predict H (unless the GPA is over 4.3 somehow).
 
\end{enumerate}

\end{document}
