\documentclass[12pt]{article}
\usepackage{url,graphicx,tabularx,array}
\usepackage[margin=1in]{geometry}
\setlength{\parskip}{1ex} %--skip lines between paragraphs
\setlength{\parindent}{0pt} %--don't indent paragraphs

\usepackage{algorithmic}
\usepackage{algorithm}
\usepackage{ amssymb }
\usepackage{ latexsym }
\usepackage{ amsmath }
\usepackage{ amsthm }
%-- Commands for header
\renewcommand{\title}[1]{\textbf{#1}\\}
\renewcommand{\line}{\begin{tabularx}{\textwidth}{X>{\raggedleft}X}\hline\\\end{tabularx}\\[-0.5cm]}
\newcommand{\leftright}[2]{\begin{tabularx}{\textwidth}{X>{\raggedleft}X}#1%
& #2\\\end{tabularx}\\[-0.5cm]}

\newtheorem{defn}{Definition}[section]
\newtheorem{conjecture}{conjecture}[section]
\newtheorem{lemma}{Lemma}[section]
\newtheorem{corollary}{Corollary}[section]
\newtheorem{question}{Question}[section]
\newtheorem{proposition}{Proposition}[section]


%\linespread{2} %-- Uncomment for Double Space
\usepackage{Sweave}
\begin{document}

\title{Homework 2: CMPS 242}
\line
\leftright{\today}{Bryan Matsuo (bmatsuo@soe.ucsc.edu) \& John St. John (jstjohn@soe.ucsc.edu)} %-- left and right positions in the header
\begin{enumerate}
\item \textbf{General Probability: }

\begin{enumerate}
\item 
\begin{Schunk}
\begin{Sinput}
> h <- matrix(data = c(0.8, 0.25, 0.5, 0.8, 0.75, 0.5, 0.2, 0.75, 
+     0.5), nrow = 3, ncol = 3, dimnames = list(c("a", "b", "c"), 
+     c("h1", "h2", "h3")), byrow = TRUE)
> print(h, type = "latex")
\end{Sinput}
\begin{Soutput}
   h1   h2  h3
a 0.8 0.25 0.5
b 0.8 0.75 0.5
c 0.2 0.75 0.5
\end{Soutput}
\begin{Sinput}
> p <- matrix(data = c(rep(1/2, 3), rep(4/10, 3), rep(1/10, 3)), 
+     nrow = 3, ncol = 3)
> priors <- apply(p * h, 1, sum)
> priors
\end{Sinput}
\begin{Soutput}
   a    b    c 
0.55 0.75 0.45 
\end{Soutput}
\end{Schunk}
\item
\begin{Schunk}
\begin{Sinput}
> obs.p <- c(1, 0, 1)
> obs.n <- c(0, 1, 0)
> likelihood.p <- apply(obs.p * h, 2, sum)
> likelihood.n <- apply(obs.n * (1 - h), 2, sum)
> likelihood <- likelihood.p + likelihood.n
> likelihood
\end{Sinput}
\begin{Soutput}
  h1   h2   h3 
1.20 1.25 1.50 
\end{Soutput}
\end{Schunk}
The hypothesis with the greatest likelihood is $h_3$.
\item


\end{enumerate}
\end{enumerate}

\end{document}
